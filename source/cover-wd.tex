%!TEX root = std.tex
%%--------------------------------------------------
%% Title page for the C++ Standard


\thispagestyle{empty}
\begingroup
\def\hd{\begin{tabular}{ll}
          \textbf{Document Number:} & P1722R2                     \\
          \textbf{Date:}            & \reldate                    \\
          \textbf{Reply to:}        & Marshall Clow               \\
                                    & mclow.lists@gmail.com
          \end{tabular}
}
\newlength{\hdwidth}
\settowidth{\hdwidth}{\hd}
\hfill\begin{minipage}{\hdwidth}\hd\end{minipage}
\endgroup

\vspace{2.5cm}
\begin{center}
\textbf{\Huge
Mandating the Standard Library:\\Clause 30 - Regular Expressions library}
\end{center}

With the adoption of P0788R3, we have a new way of specifying requirements for the
library clauses of the standard. This is one of a series of papers reformulating the
requirements into the new format. This effort was strongly influenced by the informational
paper P1369R0.

The changes in this series of papers fall into three broad categories.
\begin{itemize}
\item{Change "participate in overload resolution" wording into "Constraints" elements}
\item{Change "Requires" elements into either "Mandates" or "Expects", depending (mostly) on whether or not they can be checked at compile time.}
\item{Drive-by fixes (hopefully very few)}
\end{itemize}

This paper covers Clause 30 (Regular Expressions), and is based on N4830.

The entire clause is reproduced here, but the changes are confined to a few sections:

\begin{multicols}{2}
\begin{itemize}
\item{re.badexp}			\ref{re.badexp}
\item{re.traits}			\ref{re.traits}
\item{re.regex.construct}	\ref{re.regex.construct}
\item{re.regex.assign}		\ref{re.regex.assign}
\item{re.results.const}		\ref{re.results.const}
\item{re.results.acc}		\ref{re.results.acc}
\item{re.results.form}		\ref{re.results.form}
\item{re.alg.match}			\ref{re.alg.match}
\item{re.alg.search}		\ref{re.alg.search}
\end{itemize}
\end{multicols}

Drive-by fixes:
\begin{itemize}
\item{In [re.regex.construct] and [re.regex.assign], I strengthened some of the preconditions. "p is not a null pointer" -> "\range{p}{p+len} is a valid range"}
\item{"De-shalled" re.traits/1 (\ref{re.traits}).}
\item{reworded a bunch of constructor details in re.badexp (\ref{re.badexp}),  re.regex.construct (\ref{re.regex.construct}), and re.results.const (\ref{re.results.const}).}
\end{itemize}

Open questions:
\begin{itemize}
\item{Should I "de-shall" the entries in table 135 and 136? - Answer: can be done editorially.}
\end{itemize}

Changes from R0:
\begin{itemize}
\item{Updated to N4830}
\item{Minor spelling and layout fixes}
\end{itemize}

Changes from R1:
\begin{itemize}
\item{Changes from teleconference review 11-Oct}
\item{Use "Effects: equivalent to:" throughout [re.regex.assign]}
\item{Removed a couple more useless paragraphs in [re.regex.assign]/1 and /3}
\item{Changes from LWG review in Belfast.}
\end{itemize}

Thanks to Daniel Krügler for his advice and reviews.

\vfill
Help for the editors: The changes here can be viewed as latex sources with the following commands
\begin{verbatim}
git clone git@github.com:mclow/mandate.git
cd mandate
git diff master..chapter30 regex.tex
\end{verbatim}
