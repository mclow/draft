%!TEX root = std.tex
%%--------------------------------------------------
%% Title page for the C++ Standard


\thispagestyle{empty}
\begingroup
\def\hd{\begin{tabular}{ll}
          \textbf{Document Number:} & D0000R0                     \\
          \textbf{Date:}            & \reldate                    \\
          \textbf{Reply to:}        & Marshall Clow               \\
                                    & CppAlliance                 \\
                                    & mclow.lists@gmail.com
          \end{tabular}
}
\newlength{\hdwidth}
\settowidth{\hdwidth}{\hd}
\hfill\begin{minipage}{\hdwidth}\hd\end{minipage}
\endgroup

\vspace{2.5cm}
\begin{center}
\textbf{\Huge
Mandating the Standard Library:\\Clause 31 - Atomics library}
\end{center}

With the adoption of P0788R3, we have a new way of specifying requirements for the
library clauses of the standard. This is one of a series of papers reformulating the
requirements into the new format. This effort was strongly influenced by the informational
paper P1369R0.

The changes in this series of papers fall into four broad categories.
\begin{itemize}
\item{Change "participate in overload resolution" wording into "Constraints" elements}
\item{Change "Requires" elements into either "Mandates" or "Expects", depending (mostly) on whether or not they can be checked at compile time.}
\item{Drive-by fixes (hopefully very few)}
\end{itemize}

This paper covers Clause 31 (Atomics), and is based on N4810.

The entire clause is reproduced here, but the changes are confined to a few sections:

%\begin{multicols}{3}
%\begin{itemize}
%\item{alg.foreach}					\ref{alg.foreach}
%\item{alg.is.permutation}			\ref{alg.is.permutation}
%\item{alg.copy}						\ref{alg.copy}
%\item{alg.move}						\ref{alg.move}
%\item{alg.swap}						\ref{alg.swap}
%\item{alg.transform}				\ref{alg.transform}
%\item{alg.replace}					\ref{alg.replace}
%\item{alg.fill}						\ref{alg.fill}
%\item{alg.generate}					\ref{alg.generate}
%\item{alg.remove}					\ref{alg.remove}
%\item{alg.unique}					\ref{alg.unique}
%\item{alg.reverse}					\ref{alg.reverse}
%\item{alg.rotate}					\ref{alg.rotate}
%\item{alg.random.sample}			\ref{alg.random.sample}
%\item{alg.random.shuffle}			\ref{alg.random.shuffle}
%\item{alg.shift}					\ref{alg.shift}
%\item{sort}							\ref{sort}
%\item{stable.sort}					\ref{stable.sort}
%\item{partial.sort}					\ref{partial.sort}
%\item{partial.sort.copy}			\ref{partial.sort.copy}
%\item{alg.nth.element}				\ref{alg.nth.element}
%\item{lower.bound}					\ref{lower.bound}
%\item{upper.bound}					\ref{upper.bound}
%\item{equal.range}					\ref{equal.range}
%\item{binary.search}				\ref{binary.search}
%\item{alg.partitions}				\ref{alg.partitions}
%\item{alg.merge}					\ref{alg.merge}
%\item{includes}						\ref{includes}
%\item{set.intersection}				\ref{set.intersection}
%\item{set.difference}				\ref{set.difference}
%\item{set.symmetric.difference}		\ref{set.symmetric.difference}
%\item{push.heap}					\ref{push.heap}
%\item{pop.heap}						\ref{pop.heap}
%\item{make.heap}					\ref{make.heap}
%\item{sort.heap}					\ref{sort.heap}
%\item{alg.min.max}					\ref{alg.min.max}
%\item{alg.clamp}					\ref{alg.clamp}
%\item{alg.3way}						\ref{alg.3way}
%\item{alg.permutation.generators}	\ref{alg.permutation.generators}
%\item{accumulate}					\ref{accumulate}
%\item{reduce}						\ref{reduce}
%\item{inner.product}				\ref{inner.product}
%\item{transform.reduce}				\ref{transform.reduce}
%\item{exclusive.scan}				\ref{exclusive.scan}
%\item{inclusive.scan}				\ref{inclusive.scan}
%\item{transform.exclusive.scan}		\ref{transform.exclusive.scan}
%\item{transform.inclusive.scan}		\ref{transform.inclusive.scan}
%\item{adjacent.difference}			\ref{adjacent.difference}
%\item{numeric.iota}					\ref{numeric.iota}
%\item{numeric.ops.gcd}				\ref{numeric.ops.gcd}
%\item{numeric.ops.lcm}				\ref{numeric.ops.lcm}
%\end{itemize}
%\end{multicols}

%Drive-by fixes:
%\begin{itemize}
%\item{All of the \*_n algorithms now treat their \tcode{Size} parameter the same way. This also resolves LWG\#3213.}
%\item{As I was "de-shalling" all the Requirements, I applied the guidance at https://github.com/cplusplus/draft/wiki/Specification-Style-Guidelines\#requirements-expressed-by-concepts. }
%\end{itemize}

%Open questions:
%\begin{itemize}
%\item{alg.unique \ref{alg.unique} P7.2.3 is kind of a mess. Not quite sure how to "de-shall" it.}
%\item{alg.random.sample \ref{alg.random.sample} has a \tcode{Distance} parameter that looks like the \tcode{Size} parameter from \*_n. Should it be specified the same way?}
%\item{Where should "XX is writeable to YY" and "the expression 'Foo/bar' is valid" go? Mandates? Expects? I put them in "Expects".}
%\end{itemize}


%Changes from R0:
%\begin{itemize}
%\item{Changed a couple of the "Expects" into "Mandates".}
%\item{Changed two of "shall not be"s into "is not"s in ostreambuf.iter.cons \ref{ostreambuf.iter.cons}.}
%\item{Changed several "Constraints: The expression XXXX shall be valid and convertible to bool" to "XXXX is well-formed and convertible to bool".}
%\end{itemize}

\vfill
Help for the editors: The changes here can be viewed as latex sources with the following commands
\begin{verbatim}
git clone git@github.com:mclow/mandate.git
cd mandate
git diff master..chapter31 atomics.tex
\end{verbatim}
