%!TEX root = std.tex
%%--------------------------------------------------
%% Title page for the C++ Standard


\thispagestyle{empty}
\begingroup
\def\hd{\begin{tabular}{ll}
          \textbf{Document Number:} & P1719R2                     \\
          \textbf{Date:}            & \reldate                    \\
          \textbf{Reply to:}        & Marshall Clow               \\
                                    & mclow.lists@gmail.com
          \end{tabular}
}
\newlength{\hdwidth}
\settowidth{\hdwidth}{\hd}
\hfill\begin{minipage}{\hdwidth}\hd\end{minipage}
\endgroup

\vspace{2.5cm}
\begin{center}
\textbf{\Huge
Mandating the Standard Library:\\Clause 26 - Numerics library}
\end{center}

With the adoption of P0788R3, we have a new way of specifying requirements for the
library clauses of the standard. This is one of a series of papers reformulating the
requirements into the new format. This effort was strongly influenced by the informational
paper P1369R0.

The changes in this series of papers fall into three broad categories.
\begin{itemize}
\item{Change "participate in overload resolution" wording into "Constraints" elements}
\item{Change "Requires" elements into either "Mandates" or "Expects", depending (mostly) on whether or not they can be checked at compile time.}
\item{Drive-by fixes (hopefully very few)}
\end{itemize}

This paper covers Clause 26 (Numerics), and is based on N4830.

The entire clause is reproduced here, but the changes are confined to a few sections:

\begin{multicols}{3}
\begin{itemize}
\item{cfenv.syn}				\ref{cfenv.syn}
\item{complex.numbers}			\ref{complex.numbers}
\item{complex.members}			\ref{complex.members}
\item{complex.ops}				\ref{complex.ops}
\item{complex.value.ops}		\ref{complex.value.ops}
\item{bit.cast}					\ref{bit.cast}
\item{bit.pow.two}				\ref{bit.pow.two}
\item{rand.req.eng}				\ref{rand.req.eng}
\item{rand.req.dist}			\ref{rand.req.dist}
\item{rand.eng.lcong}			\ref{rand.eng.lcong}
\item{rand.eng.mers}			\ref{rand.eng.mers}
\item{rand.eng.sub}				\ref{rand.eng.sub}
\item{rand.device}				\ref{rand.device}
\item{rand.util.seedseq}		\ref{rand.util.seedseq}
\item{rand.dist.uni.int}		\ref{rand.dist.uni.int}
\item{rand.dist.uni.real}		\ref{rand.dist.uni.real}
\item{rand.dist.bern.bernoulli}	\ref{rand.dist.bern.bernoulli}
\item{rand.dist.bern.bin}		\ref{rand.dist.bern.bin}
\item{rand.dist.bern.geo}		\ref{rand.dist.bern.geo}
\item{rand.dist.bern.negbin}	\ref{rand.dist.bern.negbin}
\item{rand.dist.pois.poisson}	\ref{rand.dist.pois.poisson}
\item{rand.dist.pois.exp}		\ref{rand.dist.pois.exp}
\item{rand.dist.pois.gamma}		\ref{rand.dist.pois.gamma}
\item{rand.dist.pois.weibull}	\ref{rand.dist.pois.weibull}
\item{rand.dist.pois.extreme}	\ref{rand.dist.pois.extreme}
\item{rand.dist.norm.normal}	\ref{rand.dist.norm.normal}
\item{rand.dist.norm.lognormal}	\ref{rand.dist.norm.lognormal}
\item{rand.dist.norm.chisq}		\ref{rand.dist.norm.chisq}
\item{rand.dist.norm.cauchy}	\ref{rand.dist.norm.cauchy}
\item{rand.dist.norm.f}			\ref{rand.dist.norm.f}
\item{rand.dist.norm.t}			\ref{rand.dist.norm.t}
\item{rand.dist.samp.discrete}	\ref{rand.dist.samp.discrete}
\item{rand.dist.samp.pconst}	\ref{rand.dist.samp.pconst}
\item{rand.dist.samp.plinear}	\ref{rand.dist.samp.plinear}
\item{valarray.cons}			\ref{valarray.cons}
\item{valarray.assign}			\ref{valarray.assign}
\item{valarray.access}			\ref{valarray.access}
\item{valarray.unary}			\ref{valarray.unary}
\item{valarray.cassign}			\ref{valarray.cassign}
\item{valarray.members}			\ref{valarray.members}
\item{valarray.binary}			\ref{valarray.binary}
\item{valarray.comparison}		\ref{valarray.comparison}
\item{valarray.transcend}		\ref{valarray.transcend}
\end{itemize}
\end{multicols}

Drive-by fixes:
\begin{itemize}
\item{Removed several useless "Constructs an object of type XXXX" sentences, and reworked a bunch left-over "Effects" into "Remarks"}
\item{While moving "Expects" to "Mandates", changed "XXXX shall denote a type that is convertible to double" to "\tcode{is_convertible_v<XXXX, double>} is true."}
\item{While moving "Expects" to "Mandates", changed "XXXX shall be callable with a type that is convertible to double" to "\tcode{is_invocable_r_v<XXXX, double>} is true."}
\item{Changed a "points to an array of at least n elements" -> "\range{p}{p+n} is a valid range."}
\item{Changed several "Complexity" clauses from "shall not exceed" to "does not exceed".}
\item{Reworked two paragraphs in [rand.util.seedseq] to use the exposition-only variable \tcode{v}.}
\end{itemize}

Open questions:
\begin{itemize}
\item{The "xxx relation holds" formulation needs a pattern.}
\end{itemize}


Changes from R0:
\begin{itemize}
\item{Rebased on N4830}
\item{Removed several "shall"s from [cfenv.syn] (\ref{cfenv.syn}) and [complex.numbers] (\ref{complex.numbers})}
\item{Changed a "points to an array of at least N" -> "\tcode{[p, p+n) is a valid range}"}
\item{Turned a couple of sentences into bullet points in [rand.dist.samp.pconst] and prand.dist.samp.plinear].}
\item{Items discussed during the LWG telecom on 6-Sep-2019.}
\item{Reworked two paragraphs in [rand.util.seedseq] to use the exposition-only variable \tcode{v}.}
\end{itemize}

Changes from R1:
\begin{itemize}
\item{Changes from LWG review in Belfast}
\end{itemize}

Thanks to Daniel Krügler for his several rounds of review.

\vfill
Help for the editors: The changes here can be viewed as latex sources with the following commands
\begin{verbatim}
git clone git@github.com:mclow/mandate.git
cd mandate
git diff master..chapter26 numerics.tex
\end{verbatim}
